\section{Conclusions and Future Work}

In this paper we propose that by fuzzing variations of a target program generated by a mutation testing tool, it may be possible to work around some fundamental limitations of coverage-driven fuzzing.  For the most part, even when not effective, the technique proposed should be low-cost and at worst equivalent to fuzzing the target program itself for a somewhat smaller time.  Our preliminary experiments show that fuzzing mutants is trivial to implement (and applies to any fuzzer of which we are aware) and effective for improving fault detection, at least for a non-trivial benchmark target program.

Future work, in addition to the performance of full experiments to evaluate the technique, would include exploring the effectiveness of using mutation selection methods and prioritization techniques in addition to those proposed here, and applying directed greybox fuzzing \cite{AFLGo} to specifically target mutated code.  Another possibility is to use Higher Order Mutants \cite{HOM} to fuzz multiple mutants at once; however, this increases the chance that a critical mutant will be combined with a mutant that essentially destroys the program semantics, making it impossible to exploit.

\section*{Acknowledgements}
\begin{sloppypar}
  A portion of this work was
  supported by the National Science Foundation under CCF-2129446.  The
authors would also like to thank our anonymous reviewers.
\end{sloppypar}
