\documentclass[11pt]{article}

%\usepackage{helvet} % Uncomment this (while commenting the above line) to use the Helvetica font

% Margins
\topmargin=-1in
\textheight=8.5in
\oddsidemargin=-10pt
\textwidth=6.5in

%\let\raggedleft\raggedright % Pushes the date (at the top) to the left, comment this line to have the date on the right

\begin{document}
\section{Changes}
\subsection{Changes requested by Reviewer1}
\begin{itemize}
\item \textbf{Formula add $d:=$ in front of the equation.} \\
We have added this in front of the formula.

\item \textbf{$max(0.25, ...)$ should be probably $min()$, which would match the
textual description with: ``, but caps the amount added at 0.25''.}\\
Note that the textual description clearly says that we \emph{cap} the amount
added at 0.25. That is, the maximum is 0.25.
\item \textbf{Change Alternative Prioritizations.: to Alternative
Prioritizations:}\\
We have updated the punctuation..

\item \textbf{Change Full Mutant Analysis, Continuous Mutant Analysis:: to Full Mutant
Analysis, Continuous Mutant Analysis:}\\
We have updated the punctuation.

\item \textbf{Change hard-to-teach to hard-to-reach.}\\
We have fixed the spelling mistake.

\item \textbf{5555 is the wrong year in citation 20.}\\
We have fixed the year in citation.

\item \textbf{
The RQs should be refined. RQ1 resembles the overall question of whether this
method is effective. RQ2 and RQ3 are sub-question that try to answer this for
changes along different dimensions in the solutions space. Nevertheless, those
questions are important. A question in a similar direction would be: How much of
a budget should be spent on the fuzzing of the mutants?
RQ4 is not clear to me.
In addition to the existing questions, I think the submission could be
strengthened by answering the question of what kind of mutations have the
strongest impact. Do specific mutations have almost no effect? Or are some
counter-productive?
}
We have reformulated all the RQs.


\end{itemize}
\subsection{Changes requested by Reviewer3}
\begin{itemize}
\item \textbf{Less than 1\%" does not seem correct. 144 / 2000 = 7.2\%}\\
We have changed this to less than 10\%.

\item \textbf{Finally, for especially critical fuzzing targets, especially
those" : the first use of "especially" is superfluous.}\\
Removed the first use.

\item \textbf{In fact, a CI-style continuous fuzzing effort could in principle
alternative fuzzing the target program with a rolling sequence of mutants --
something is missing here.}\\
Fixed it to: In fact, a CI-style continuous fuzzing effort could in principle be
an alternative to fuzzing the target program with a rolling sequence of mutants

\item \textbf{"(ro at least those that ever generated useful inputs)" : Change
ro to to} \\
Fixed the spelling mistake.

\item \textbf{demarction to demarcation}\\
Fixed the spelling mistake.

\item \textbf{While the total computing resources required to fuzz the same
number of mutants are constant, that one approach is (embarrassingly) parallel
is a significant advantage in modern multicore contexts." : The second part
sounds a bit overcomplicated,}

Fixed it to: While the total computing resources required to fuzz the same
number of mutants are constant, an approach that is (embarrassingly) parallel
has a significant advantage in modern multicore contexts.                                          

\item \textbf{checksum based to checksum-based}\\
Fixed the spelling mistake.

\item \textbf{three of the mutant-baed efforts to three of the mutant-based
efforts}\\
Fixed the spelling mistake.

\end{itemize}


\end{document}
