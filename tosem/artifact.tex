%%
%% This is file `sample-manuscript.tex',
%% generated with the docstrip utility.
%%
%% The original source files were:
%%
%% samples.dtx  (with options: `manuscript')
%% 
%% IMPORTANT NOTICE:
%% 
%% For the copyright see the source file.
%% 
%% Any modified versions of this file must be renamed
%% with new filenames distinct from sample-manuscript.tex.
%% 
%% For distribution of the original source see the terms
%% for copying and modification in the file samples.dtx.
%% 
%% This generated file may be distributed as long as the
%% original source files, as listed above, are part of the
%% same distribution. (The sources need not necessarily be
%% in the same archive or directory.)
%%
%%
%% Commands for TeXCount
%TC:macro \cite [option:text,text]
%TC:macro \citep [option:text,text]
%TC:macro \citet [option:text,text]
%TC:envir table 0 1
%TC:envir table* 0 1
%TC:envir tabular [ignore] word
%TC:envir displaymath 0 word
%TC:envir math 0 word
%TC:envir comment 0 0
%%
%%
%% The first command in your LaTeX source must be the \documentclass command.
\documentclass[manuscript,screen,review]{acmart}

%%
%% \BibTeX command to typeset BibTeX logo in the docs
\AtBeginDocument{%
  \providecommand\BibTeX{{%
    Bib\TeX}}}

%% Rights management information.  This information is sent to you
%% when you complete the rights form.  These commands have SAMPLE
%% values in them; it is your responsibility as an author to replace
%% the commands and values with those provided to you when you
%% complete the rights form.
\setcopyright{acmcopyright}
\copyrightyear{2018}
\acmYear{2018}
\acmDOI{XXXXXXX.XXXXXXX}

%% These commands are for a PROCEEDINGS abstract or paper.
\acmConference[Conference acronym 'XX]{Make sure to enter the correct
  conference title from your rights confirmation emai}{June 03--05,
  2018}{Woodstock, NY}
\acmPrice{15.00}
\acmISBN{978-1-4503-XXXX-X/18/06}


%%
%% Submission ID.
%% Use this when submitting an article to a sponsored event. You'll
%% receive a unique submission ID from the organizers
%% of the event, and this ID should be used as the parameter to this command.
%%\acmSubmissionID{123-A56-BU3}

%%
%% For managing citations, it is recommended to use bibliography
%% files in BibTeX format.
%%
%% You can then either use BibTeX with the ACM-Reference-Format style,
%% or BibLaTeX with the acmnumeric or acmauthoryear sytles, that include
%% support for advanced citation of software artefact from the
%% biblatex-software package, also separately available on CTAN.
%%
%% Look at the sample-*-biblatex.tex files for templates showcasing
%% the biblatex styles.
%%

%%
%% The majority of ACM publications use numbered citations and
%% references.  The command \citestyle{authoryear} switches to the
%% "author year" style.
%%
%% If you are preparing content for an event
%% sponsored by ACM SIGGRAPH, you must use the "author year" style of
%% citations and references.
%% Uncommenting
%% the next command will enable that style.
%%\citestyle{acmauthoryear}

\author{Alex Groce}
\affiliation{\institution{Northern Arizona University}\country{United States}}

\author{Goutamkumar Tulajappa Kalburgi}
\affiliation{\institution{Northern Arizona University}\country{United States}}

\author{Claire Le Goues}
\affiliation{\institution{Carnegie Mellon University}\country{United States}}

\author{Kush Jain}
\affiliation{\institution{Carnegie Mellon University}\country{United States}}

\author{Rahul Gopinath}
\affiliation{\institution{University of Sydney}\country{Australia}}

\usepackage{code}
\usepackage{url}

%%
%% By default, the full list of authors will be used in the page
%% headers. Often, this list is too long, and will overlap
%% other information printed in the page headers. This command allows
%% the author to define a more concise list
%% of authors' names for this purpose.
\renewcommand{\shortauthors}{Alex Groce, Kush Jain, Rijnard van Tonder, Goutamkumar Tulajappa Kalburgi, Claire Le Goues}

%%
%% end of the preamble, start of the body of the document source.
\begin{document}

%%
%% The "title" command has an optional parameter,
%% allowing the author to define a "short title" to be used in page headers.
\title{RCR Artifact Report: First, Fuzz the Mutants}

%%
%% The "author" command and its associated commands are used to define
%% the authors and their affiliations.
%% Of note is the shared affiliation of the first two authors, and the
%% "authornote" and "authornotemark" commands
%% used to denote shared contribution to the research.

%%
%% By default, the full list of authors will be used in the page
%% headers. Often, this list is too long, and will overlap
%% other information printed in the page headers. This command allows
%% the author to define a more concise list
%% of authors' names for this purpose.
\renewcommand{\shortauthors}{Groce et al.}

%%
%% The abstract is a short summary of the work to be presented in the
%% article.
\begin{abstract}
The artifact present in this report is important to the paper, as it allows anyone to reproduce the results that
we state, along with access all of the source code that we added for our first, fuzz the mutants approach for both 
aflplusplus and honggfuzz. The process to reproduce our results and explenation of the artifacts provided are all present
in this paper. Additionally, the authors have integrated with fuzzbench, a popular fuzzing benchmark suite, to allow for others 
to easily run experiments using the fuzzers present in this paper over a large variety of fuzzing benchmarks.
\end{abstract}

%%
%% The code below is generated by the tool at http://dl.acm.org/ccs.cfm.
%% Please copy and paste the code instead of the example below.
%%

\begin{CCSXML}
<ccs2012>
<concept>
<concept_id>10011007.10010940.10010992.10010998.10011001</concept_id>
<concept_desc>Software and its engineering~Dynamic analysis</concept_desc>
<concept_significance>500</concept_significance>
</concept>
<concept>
<concept_id>10011007.10011074.10011099.10011102.10011103</concept_id>
<concept_desc>Software and its engineering~Software testing and debugging</concept_desc>
<concept_significance>500</concept_significance>
</concept>
</ccs2012>
\end{CCSXML}

\ccsdesc[500]{Software and its engineering~Dynamic analysis}
\ccsdesc[500]{Software and its engineering~Software testing and debugging}

\keywords{fuzzing, mutation testing}

%%
%% This command processes the author and affiliation and title
%% information and builds the first part of the formatted document.
\maketitle

\subsection{Paper}
In our paper, First, Fuzz the Mutants we present a novel approach to improving the overall performance of traditional fuzzing loops, where 
we apply mutation analysis to programs that are being fuzzed and perform fuzzing on these binaries first. We then aggregate the outputs of
fuzzing these binaries and use them as an input corpus to fuzzing the original program. We ask the following research questions, which we answer
through a combination of large scale experiments on fuzzbench and small scale experiments on TFuzz benchmarks and fuzzgoat:
\begin{itemize}
    \item {\bf RQ1}: Can replacing time spent fuzzing a target program with time spent fuzzing mutants of the target program, under at least one configuration, improve
    the effectiveness of fuzzing on average over a variety of target programs?
    \item {\bf RQ2:} Does using prioritization improve the effectiveness of fuzzing with mutants?  If so, which prioritizations perform best?
    \item {\bf RQ3:} How do non-cumulative (parallel) and cumulative (sequential) mutant fuzzing compare?
    \item {\bf RQ4:} How does impact of using mutants vary with the fraction of the fuzzing budget devoted to fuzzing mutants, given a good configuration based on results from {\bf RQ2} and {\bf RQ3}?
    \item {\bf RQ5:} Are certain kinds of mutants generally more useful for fuzzing, or generally not useful?  Are these patterns the same or different for new path discovery and novel bug detection?
\end{itemize}

For RQ1, we perform a series of large scale experiments on fuzzbench, where we run our mutation guided fuzzing approach on aflplusplus and honggfuzz over all fuzzbench benchmarks. Fuzzbench runs our
fuzzers for 24 hours, 20 times each. We find that in general, our fuzzers perform similarly to vanilla aflplusplus and vanilla honggfuzz, with the majority of the benchmarks showing all differences between
our fuzzers and these vanilla fuzzers within a 95\% confidence interval. However, for certain benchmarks with fewer files to mutate, such as bloaty-fuzz-target, our approach performs far better than baselines
with 95\% confidence intervals not overlapping at all. For sqllite we outperform baselines significantly at first, however as time progresses vanilla fuzzers catch up with our approach.

For RQ2, RQ3, and RQ4 we also use our fuzzbench setup, simply running variants of our fuzzer on the benchmark suite. TODO: Add results here...

For RQ5, we were unable to use fuzzbench due to us not being able to access build files that generated the mutants, as these were part of the build docker container that gets removed immediately after the build
phase of fuzzbench finishes. However, we were able to answer this question through smaller scale experiments on fuzzgoat and TFuzz benchmarks, where we wrote simple code to apply our technique. This code
is packaged as part of the artifact along with fuzzbench code.

\subsection{Artifact}

\begin{table}
    \caption{Overview of All Artifacts, Link and License.}
    \begin{tabular}{lll}
        \toprule
        \bf Description                   & \bf Link  & \bf License \\
        \midrule
        aflplusplus\_um\_random   & \url{https://github.com/google/fuzzbench/tree/master/fuzzers/aflplusplus_um_random} & Apache 2.0       \\
        aflplusplus\_um\_prioritize   & \url{https://github.com/google/fuzzbench/tree/master/fuzzers/aflplusplus_um_prioritize} & Apache 2.0       \\
        aflplusplus\_um\_parallel   & \url{https://github.com/google/fuzzbench/tree/master/fuzzers/aflplusplus_um_parallel} & Apache 2.0       \\
        honggfuzz\_um\_random   & \url{https://github.com/google/fuzzbench/tree/master/fuzzers/honggfuzz_um_random} & Apache 2.0       \\
        honggfuzz\_um\_prioritize   & \url{https://github.com/google/fuzzbench/tree/master/fuzzers/honggfuzz_um_prioritize} & Apache 2.0       \\
        honggfuzz\_um\_parallel   & \url{https://github.com/google/fuzzbench/tree/master/fuzzers/honggfuzz_um_parallel} & Apache 2.0       \\
        small scale experiments   & \url{https://github.com/agroce/fuzzing22report/tree/master/code} & Apache 2.0       \\
        \bottomrule
    \end{tabular}
    \label{tab:artifacts}
\end{table}

We detail all of the components and steps needed to reproduce our experiments in this section. This includes the code and steps needed to run fuzzbench experiments, along with code to reproduce our smaller
scale experiments on TFuzz and fuzzgoat benchmarks. 

Table \ref{tab:artifacts} shows all code that we have released and their locations. All code is licensed under Apache 2.0 to ensure ease of use both commercially and academically. This paper
did not include any evidence collection or generation phase, with all of our evaluation benchmarks already preexisting.

\subsubsection{Prerequisites}
For our fuzzbench experiments, there are no prerequisites, as you can simply request an experiment that will be run on Google Cloud Servers and then receive a report detailing our fuzzer in comparison
against other fuzzers. For the small scale experiments, we ran them on a Linux machine, with TODO: add specs here.

\subsubsection{Steps to Reproduce}
To reproduce our results for RQ1, RQ2, RQ3 and RQ4, simply head to fuzzbench's main repository \url{https://github.com/google/fuzzbench} and edit the following file 
\url{https://github.com/google/fuzzbench/blob/master/service/experiment-requests.yaml}. This file lists all experiments to be run on Google Cloud Servers. Below are snippets to
add to the file for each RQ:

RQ1, RQ2, RQ3, RQ4:
\begin{code}
\- experiment: <YOUR NAME HERE>
    description: "<YOUR DESCRIPTION HERE>"
    fuzzers:
      \- aflplusplus
      \- aflplusplus\_um\_random
      \- aflplusplus\_um\_prioritize
      \- aflplusplus\_um\_random\_75
      \- aflplusplus\_um\_prioritize\_75
      \- aflplusplus\_um\_parallel
      \- honggfuzz
      \- honggfuzz\_um\_random
      \- honggfuzz\_um\_prioritize
      \- honggfuzz\_um\_random\_75
      \- honggfuzz\_um\_prioritize\_75
      \- honggfuzz\_um\_parallel
\end{code}

For analysis on this fuzzbench data, reports are available at \url{https://www.fuzzbench.com/reports/experimental/}. These reports show a comparison of our fuzzer against existing 
fuzzers along with 95\% confidence intervals needed to determine if the differences we see are statistically significant.

For our smaller scale experiments... TODO: add steps here

\end{document}
\endinput
%%
%% End of file `sample-manuscript.tex'.
