%%
%% This is file `sample-manuscript.tex',
%% generated with the docstrip utility.
%%
%% The original source files were:
%%
%% samples.dtx  (with options: `manuscript')
%% 
%% IMPORTANT NOTICE:
%% 
%% For the copyright see the source file.
%% 
%% Any modified versions of this file must be renamed
%% with new filenames distinct from sample-manuscript.tex.
%% 
%% For distribution of the original source see the terms
%% for copying and modification in the file samples.dtx.
%% 
%% This generated file may be distributed as long as the
%% original source files, as listed above, are part of the
%% same distribution. (The sources need not necessarily be
%% in the same archive or directory.)
%%
%%
%% Commands for TeXCount
%TC:macro \cite [option:text,text]
%TC:macro \citep [option:text,text]
%TC:macro \citet [option:text,text]
%TC:envir table 0 1
%TC:envir table* 0 1
%TC:envir tabular [ignore] word
%TC:envir displaymath 0 word
%TC:envir math 0 word
%TC:envir comment 0 0
%%
%%
%% The first command in your LaTeX source must be the \documentclass command.
\documentclass[manuscript,screen,review]{acmart}

%%
%% \BibTeX command to typeset BibTeX logo in the docs
\AtBeginDocument{%
  \providecommand\BibTeX{{%
    Bib\TeX}}}

%% Rights management information.  This information is sent to you
%% when you complete the rights form.  These commands have SAMPLE
%% values in them; it is your responsibility as an author to replace
%% the commands and values with those provided to you when you
%% complete the rights form.
\setcopyright{acmcopyright}
\copyrightyear{2018}
\acmYear{2018}
\acmDOI{XXXXXXX.XXXXXXX}

%% These commands are for a PROCEEDINGS abstract or paper.
\acmConference[Conference acronym 'XX]{Make sure to enter the correct
  conference title from your rights confirmation emai}{June 03--05,
  2018}{Woodstock, NY}
\acmPrice{15.00}
\acmISBN{978-1-4503-XXXX-X/18/06}


%%
%% Submission ID.
%% Use this when submitting an article to a sponsored event. You'll
%% receive a unique submission ID from the organizers
%% of the event, and this ID should be used as the parameter to this command.
%%\acmSubmissionID{123-A56-BU3}

%%
%% For managing citations, it is recommended to use bibliography
%% files in BibTeX format.
%%
%% You can then either use BibTeX with the ACM-Reference-Format style,
%% or BibLaTeX with the acmnumeric or acmauthoryear sytles, that include
%% support for advanced citation of software artefact from the
%% biblatex-software package, also separately available on CTAN.
%%
%% Look at the sample-*-biblatex.tex files for templates showcasing
%% the biblatex styles.
%%

%%
%% The majority of ACM publications use numbered citations and
%% references.  The command \citestyle{authoryear} switches to the
%% "author year" style.
%%
%% If you are preparing content for an event
%% sponsored by ACM SIGGRAPH, you must use the "author year" style of
%% citations and references.
%% Uncommenting
%% the next command will enable that style.
%%\citestyle{acmauthoryear}

\author{Alex Groce}
\affiliation{\institution{Northern Arizona University}\country{United States}}

\author{Kush Jain}
\affiliation{\institution{Carnegie Mellon University}\country{United States}}

\author{Goutamkumar Tulajappa Kalburgi}
\affiliation{\institution{Northern Arizona University}\country{United States}}

\author{Claire Le Goues}
\affiliation{\institution{Carnegie Mellon University}\country{United States}}

\author{Rahul Gopinath}
\affiliation{\institution{University of Sydney}\country{Australia}}

\usepackage{code}
\usepackage{url}

%%
%% By default, the full list of authors will be used in the page
%% headers. Often, this list is too long, and will overlap
%% other information printed in the page headers. This command allows
%% the author to define a more concise list
%% of authors' names for this purpose.


%%
%% end of the preamble, start of the body of the document source.
\begin{document}

%%
%% The "title" command has an optional parameter,
%% allowing the author to define a "short title" to be used in page headers.
\title{Registered Report: First, Fuzz the Mutants - RCR Report}

%%
%% The "author" command and its associated commands are used to define
%% the authors and their affiliations.
%% Of note is the shared affiliation of the first two authors, and the
%% "authornote" and "authornotemark" commands
%% used to denote shared contribution to the research.

%%
%% By default, the full list of authors will be used in the page
%% headers. Often, this list is too long, and will overlap
%% other information printed in the page headers. This command allows
%% the author to define a more concise list
%% of authors' names for this purpose.
\renewcommand{\shortauthors}{Groce et al.}

%%
%% The abstract is a short summary of the work to be presented in the
%% article.
\begin{abstract}
The artifacts for this paper include the source code we contributed to
Google's FuzzBench platform, which executed the experiments reported
in the paper.  The full FuzzBench report on our experiments is also
included, and a standalone copy of FuzzBench itself to allow our
experiments to be duplicated even if the FuzzBench service is no
longer offered by Google.
\end{abstract}

%%
%% The code below is generated by the tool at http://dl.acm.org/ccs.cfm.
%% Please copy and paste the code instead of the example below.
%%

\begin{CCSXML}
<ccs2012>
<concept>
<concept_id>10011007.10010940.10010992.10010998.10011001</concept_id>
<concept_desc>Software and its engineering~Dynamic analysis</concept_desc>
<concept_significance>500</concept_significance>
</concept>
<concept>
<concept_id>10011007.10011074.10011099.10011102.10011103</concept_id>
<concept_desc>Software and its engineering~Software testing and debugging</concept_desc>
<concept_significance>500</concept_significance>
</concept>
</ccs2012>
\end{CCSXML}

\ccsdesc[500]{Software and its engineering~Dynamic analysis}
\ccsdesc[500]{Software and its engineering~Software testing and debugging}

\keywords{fuzzing, mutation testing}

%%
%% This command processes the author and affiliation and title
%% information and builds the first part of the formatted document.
\maketitle

\section{Overview}

\subsection{Paper}

\subsubsection{Summary.} Our paper proposes that fuzzing can in many cases be improved by
simply devoting some portion of the fuzzing budget to \emph{mutants}
of the program to be fuzzed.  We suggest our reasons for this
hypothesis, and in the original registered report showed promising
results for a small benchmark example.

\subsubsection{Key claims:}  Our key claim is that devoting 50\% of
the budget for fuzzing a program to fuzzing source code mutants of
that program can improve fuzzing outcomes.  A secondary claim is that
prioritizing mutants by some reasonable scheme (including a kind of
stratified sampling) will perform better than randomly selecting
mutants, but that either will be effective.

Our full set of research questions is;

\begin{itemize}
    \item {\bf RQ1}: Can replacing time spent fuzzing a target program with time spent fuzzing mutants of the target program, under at least one configuration, improve
    the effectiveness of fuzzing on average over a variety of target programs?
    \item {\bf RQ2:} Does using prioritization improve the effectiveness of fuzzing with mutants?  If so, which prioritizations perform best?
    \item {\bf RQ3:} How do non-cumulative (parallel) and cumulative (sequential) mutant fuzzing compare?
    \item {\bf RQ4:} How does impact of using mutants vary with the fraction of the fuzzing budget devoted to fuzzing mutants, given a good configuration based on results from {\bf RQ2} and {\bf RQ3}?
    \item {\bf RQ5:} Are certain kinds of mutants generally more useful for fuzzing, or generally not useful?  Are these patterns the same or different for new path discovery and novel bug detection?
    \end{itemize}

    Some of these are not answered in this submission, due to problems
    with FuzzBench discussed.  However, the core questions, RQ1 and
    RQ2 are both substantially addressed and confirm our hypotheses.

\subsubsection{Key results:} Using the two standard measures of fuzzer
effectiveness, FuzzBench experiments over a large set of benchmarks
substantiate our claims.  Using prioritized mutants as targets for
half of the fuzzing budget of AFLplusplus outperforms baseline
AFLplusplus and all other fuzzers applied.  Moreover, even random
selection of mutants does well by various measures, and increases
branch coverage by a large, statisticaly significant, amount for
multiple benchmarks.


\subsection{Artifact}

\begin{table}
  {\scriptsize
    \begin{tabular}{lll}
        \toprule
        \bf Description                   & \bf Link  & \bf License \\
        \midrule
        aflplusplus\_um\_random   & \url{https://github.com/google/fuzzbench/tree/master/fuzzers/aflplusplus_um_random} & Apache 2.0       \\
        aflplusplus\_um\_prioritize   & \url{https://github.com/google/fuzzbench/tree/master/fuzzers/aflplusplus_um_prioritize} & Apache 2.0       \\
        aflplusplus\_um\_parallel   & \url{https://github.com/google/fuzzbench/tree/master/fuzzers/aflplusplus_um_parallel} & Apache 2.0       \\
        honggfuzz\_um\_random   & \url{https://github.com/google/fuzzbench/tree/master/fuzzers/honggfuzz_um_random} & Apache 2.0       \\
        honggfuzz\_um\_prioritize   & \url{https://github.com/google/fuzzbench/tree/master/fuzzers/honggfuzz_um_prioritize} & Apache 2.0       \\
        honggfuzz\_um\_parallel   &
                                    \url{https://github.com/google/fuzzbench/tree/master/fuzzers/honggfuzz_um_parallel}
                                                      & Apache 2.0
      \\
      \hline
       All FuzzBench code &
                                                              \url{https://github.com/agroce/fuzzing22report/tree/master/fuzzbench_code}
                                                      & Apache 2.0 \\
      \hline
      FuzzBench merged report &
                         \url{https://github.com/agroce/fuzzing22report/tree/master/fuzzbench_report_10_17/report}
                                                      & Apache 2.0 \\
                                                      & Apache 2.0 \\
      \hline
      All of the above code and data &
                         https://doi.org/10.6084/m9.figshare.21356328
                                                      & Apache 2.0 \\
        \bottomrule
    \end{tabular}
    }
    \caption{Overview of All Artifacts, Links and Licenses.}    
    \label{tab:artifacts}
  \end{table}

Our \emph{evidence} is in the form of a FuzzBench report generated by merging
the results from a set of individual experiments (these are also just
fuzzer run data).  For convenience we
have included a DOI to a collection of all our code and data.  FuzzBench uses its own
data analysis tools, primarily from pandas and scikit, to our knowledge.
  
\section{Prerequisites and Requirements}


The core claims of our paper are based on FuzzBench
\cite{metzman2021fuzzbench}
(\url{https://google.github.io/fuzzbench/}) experiments.  There is no
prerequisite for performing experiments in this case; any experiment
submitted to Google's Cloud Servers as documented on the FuzzBench site
can include our fuzzers and compare them to other fuzzers.
To duplicate our experiments, some adjustments to FuzzBench build
timeouts may be required, as noted in the full paper.  The individual experiment
runs (if exact duplication is desired) are: \url{https://fuzzbench.com/reports/experimental/2022-10-13-um-final-1/index.html}, \url{https://fuzzbench.com/reports/experimental/2022-10-13-um-final-2/index.html}, \url{https://fuzzbench.com/reports/experimental/2022-10-13-um-final-4/index.html}, and \url{https://fuzzbench.com/reports/experimental/2022-10-13-um-final-5/index.html}.

\section{Setup}

As noted, the experiments used to substantiate our claims are based on
adding fuzzers to FuzzBench.  Links above to our individual
experiments should enable anyone to request a duplicate set of
experiments.  The FuzzBench team can assist in merging multiple
experiments into one run.

\section{Steps to Reproduce}

For our primary results, as noted, simply request a set of FuzzBench
experiments as documented
(\url{https://google.github.io/fuzzbench/}).  In the future we may
make newer versions of our fuzzers available that avoid FuzzBench
build problems, but we plan to leave our original fuzzers in place as
artifacts for this paper.  It is also possible to run local
experiments using FuzzBench fuzzers
(\url{https://google.github.io/fuzzbench/running-a-local-experiment}),
and we include the full FuzzBench release code and instructions for
using it standalone as well.

\bibliographystyle{ACM-Reference-Format}

\bibliography{bibliography}
\end{document}
\endinput
%%
%% End of file `sample-manuscript.tex'.
