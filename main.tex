%% Modified for NDSS 2021 by DB on 2020/12/23
%%
%% bare_conf.tex
%% V1.3
%% 2007/01/11
%% by Michael Shell
%% See:
%% http://www.michaelshell.org/
%% for current contact information.
%%
%% This is a skeleton file demonstrating the use of IEEEtran.cls
%% (requires IEEEtran.cls version 1.7 or later) with an IEEE conference paper.
%%
%% Support sites:
%% http://www.michaelshell.org/tex/ieeetran/
%% http://www.ctan.org/tex-archive/macros/latex/contrib/IEEEtran/
%% and
%% http://www.ieee.org/

%%*************************************************************************
%% Legal Notice:
%% This code is offered as-is without any warranty either expressed or
%% implied; without even the implied warranty of MERCHANTABILITY or
%% FITNESS FOR A PARTICULAR PURPOSE! 
%% User assumes all risk.
%% In no event shall IEEE or any contributor to this code be liable for
%% any damages or losses, including, but not limited to, incidental,
%% consequential, or any other damages, resulting from the use or misuse
%% of any information contained here.
%%
%% All comments are the opinions of their respective authors and are not
%% necessarily endorsed by the IEEE.
%%
%% This work is distributed under the LaTeX Project Public License (LPPL)
%% ( http://www.latex-project.org/ ) version 1.3, and may be freely used,
%% distributed and modified. A copy of the LPPL, version 1.3, is included
%% in the base LaTeX documentation of all distributions of LaTeX released
%% 2003/12/01 or later.
%% Retain all contribution notices and credits.
%% ** Modified files should be clearly indicated as such, including  **
%% ** renaming them and changing author support contact information. **
%%
%% File list of work: IEEEtran.cls, IEEEtran_HOWTO.pdf, bare_adv.tex,
%%                    bare_conf.tex, bare_jrnl.tex, bare_jrnl_compsoc.tex
%%*************************************************************************

% *** Authors should verify (and, if needed, correct) their LaTeX system  ***
% *** with the testflow diagnostic prior to trusting their LaTeX platform ***
% *** with production work. IEEE's font choices can trigger bugs that do  ***
% *** not appear when using other class files.                            ***
% The testflow support page is at:
% http://www.michaelshell.org/tex/testflow/



% Note that the a4paper option is mainly intended so that authors in
% countries using A4 can easily print to A4 and see how their papers will
% look in print - the typesetting of the document will not typically be
% affected with changes in paper size (but the bottom and side margins will).
% Use the testflow package mentioned above to verify correct handling of
% both paper sizes by the user's LaTeX system.
%
% Also note that the "draftcls" or "draftclsnofoot", not "draft", option
% should be used if it is desired that the figures are to be displayed in
% draft mode.
%
\documentclass[conference]{IEEEtran}
% Add the compsoc option for Computer Society conferences.
%
% If IEEEtran.cls has not been installed into the LaTeX system files,
% manually specify the path to it like:
% \documentclass[conference]{../sty/IEEEtran}

\pagestyle{plain}


% Some very useful LaTeX packages include:
% (uncomment the ones you want to load)


% *** MISC UTILITY PACKAGES ***
%
%\usepackage{ifpdf}
% Heiko Oberdiek's ifpdf.sty is very useful if you need conditional
% compilation based on whether the output is pdf or dvi.
% usage:
% \ifpdf
%   % pdf code
% \else
%   % dvi code
% \fi
% The latest version of ifpdf.sty can be obtained from:
% http://www.ctan.org/tex-archive/macros/latex/contrib/oberdiek/
% Also, note that IEEEtran.cls V1.7 and later provides a builtin
% \ifCLASSINFOpdf conditional that works the same way.
% When switching from latex to pdflatex and vice-versa, the compiler may
% have to be run twice to clear warning/error messages.






% *** CITATION PACKAGES ***
%
%\usepackage{cite}
% cite.sty was written by Donald Arseneau
% V1.6 and later of IEEEtran pre-defines the format of the cite.sty package
% \cite{} output to follow that of IEEE. Loading the cite package will
% result in citation numbers being automatically sorted and properly
% "compressed/ranged". e.g., [1], [9], [2], [7], [5], [6] without using
% cite.sty will become [1], [2], [5]--[7], [9] using cite.sty. cite.sty's
% \cite will automatically add leading space, if needed. Use cite.sty's
% noadjust option (cite.sty V3.8 and later) if you want to turn this off.
% cite.sty is already installed on most LaTeX systems. Be sure and use
% version 4.0 (2003-05-27) and later if using hyperref.sty. cite.sty does
% not currently provide for hyperlinked citations.
% The latest version can be obtained at:
% http://www.ctan.org/tex-archive/macros/latex/contrib/cite/
% The documentation is contained in the cite.sty file itself.






% *** GRAPHICS RELATED PACKAGES ***
%
\ifCLASSINFOpdf
  % \usepackage[pdftex]{graphicx}
  % declare the path(s) where your graphic files are
  % \graphicspath{{../pdf/}{../jpeg/}}
  % and their extensions so you won't have to specify these with
  % every instance of \includegraphics
  % \DeclareGraphicsExtensions{.pdf,.jpeg,.png}
\else
  % or other class option (dvipsone, dvipdf, if not using dvips). graphicx
  % will default to the driver specified in the system graphics.cfg if no
  % driver is specified.
  % \usepackage[dvips]{graphicx}
  % declare the path(s) where your graphic files are
  % \graphicspath{{../eps/}}
  % and their extensions so you won't have to specify these with
  % every instance of \includegraphics
  % \DeclareGraphicsExtensions{.eps}
\fi
% graphicx was written by David Carlisle and Sebastian Rahtz. It is
% required if you want graphics, photos, etc. graphicx.sty is already
% installed on most LaTeX systems. The latest version and documentation can
% be obtained at: 
% http://www.ctan.org/tex-archive/macros/latex/required/graphics/
% Another good source of documentation is "Using Imported Graphics in
% LaTeX2e" by Keith Reckdahl which can be found as epslatex.ps or
% epslatex.pdf at: http://www.ctan.org/tex-archive/info/
%
% latex, and pdflatex in dvi mode, support graphics in encapsulated
% postscript (.eps) format. pdflatex in pdf mode supports graphics
% in .pdf, .jpeg, .png and .mps (metapost) formats. Users should ensure
% that all non-photo figures use a vector format (.eps, .pdf, .mps) and
% not a bitmapped formats (.jpeg, .png). IEEE frowns on bitmapped formats
% which can result in "jaggedy"/blurry rendering of lines and letters as
% well as large increases in file sizes.
%
% You can find documentation about the pdfTeX application at:
% http://www.tug.org/applications/pdftex





% *** MATH PACKAGES ***
%
%\usepackage[cmex10]{amsmath}
% A popular package from the American Mathematical Society that provides
% many useful and powerful commands for dealing with mathematics. If using
% it, be sure to load this package with the cmex10 option to ensure that
% only type 1 fonts will utilized at all point sizes. Without this option,
% it is possible that some math symbols, particularly those within
% footnotes, will be rendered in bitmap form which will result in a
% document that can not be IEEE Xplore compliant!
%
% Also, note that the amsmath package sets \interdisplaylinepenalty to 10000
% thus preventing page breaks from occurring within multiline equations. Use:
%\interdisplaylinepenalty=2500
% after loading amsmath to restore such page breaks as IEEEtran.cls normally
% does. amsmath.sty is already installed on most LaTeX systems. The latest
% version and documentation can be obtained at:
% http://www.ctan.org/tex-archive/macros/latex/required/amslatex/math/





% *** SPECIALIZED LIST PACKAGES ***
%
%\usepackage{algorithmic}
% algorithmic.sty was written by Peter Williams and Rogerio Brito.
% This package provides an algorithmic environment fo describing algorithms.
% You can use the algorithmic environment in-text or within a figure
% environment to provide for a floating algorithm. Do NOT use the algorithm
% floating environment provided by algorithm.sty (by the same authors) or
% algorithm2e.sty (by Christophe Fiorio) as IEEE does not use dedicated
% algorithm float types and packages that provide these will not provide
% correct IEEE style captions. The latest version and documentation of
% algorithmic.sty can be obtained at:
% http://www.ctan.org/tex-archive/macros/latex/contrib/algorithms/
% There is also a support site at:
% http://algorithms.berlios.de/index.html
% Also of interest may be the (relatively newer and more customizable)
% algorithmicx.sty package by Szasz Janos:
% http://www.ctan.org/tex-archive/macros/latex/contrib/algorithmicx/




% *** ALIGNMENT PACKAGES ***
%
%\usepackage{array}
% Frank Mittelbach's and David Carlisle's array.sty patches and improves
% the standard LaTeX2e array and tabular environments to provide better
% appearance and additional user controls. As the default LaTeX2e table
% generation code is lacking to the point of almost being broken with
% respect to the quality of the end results, all users are strongly
% advised to use an enhanced (at the very least that provided by array.sty)
% set of table tools. array.sty is already installed on most systems. The
% latest version and documentation can be obtained at:
% http://www.ctan.org/tex-archive/macros/latex/required/tools/


%\usepackage{mdwmath}
%\usepackage{mdwtab}
% Also highly recommended is Mark Wooding's extremely powerful MDW tools,
% especially mdwmath.sty and mdwtab.sty which are used to format equations
% and tables, respectively. The MDWtools set is already installed on most
% LaTeX systems. The lastest version and documentation is available at:
% http://www.ctan.org/tex-archive/macros/latex/contrib/mdwtools/


% IEEEtran contains the IEEEeqnarray family of commands that can be used to
% generate multiline equations as well as matrices, tables, etc., of high
% quality.


%\usepackage{eqparbox}
% Also of notable interest is Scott Pakin's eqparbox package for creating
% (automatically sized) equal width boxes - aka "natural width parboxes".
% Available at:
% http://www.ctan.org/tex-archive/macros/latex/contrib/eqparbox/





% *** SUBFIGURE PACKAGES ***
%\usepackage[tight,footnotesize]{subfigure}
% subfigure.sty was written by Steven Douglas Cochran. This package makes it
% easy to put subfigures in your figures. e.g., "Figure 1a and 1b". For IEEE
% work, it is a good idea to load it with the tight package option to reduce
% the amount of white space around the subfigures. subfigure.sty is already
% installed on most LaTeX systems. The latest version and documentation can
% be obtained at:
% http://www.ctan.org/tex-archive/obsolete/macros/latex/contrib/subfigure/
% subfigure.sty has been superceeded by subfig.sty.



%\usepackage[caption=false]{caption}
%\usepackage[font=footnotesize]{subfig}
% subfig.sty, also written by Steven Douglas Cochran, is the modern
% replacement for subfigure.sty. However, subfig.sty requires and
% automatically loads Axel Sommerfeldt's caption.sty which will override
% IEEEtran.cls handling of captions and this will result in nonIEEE style
% figure/table captions. To prevent this problem, be sure and preload
% caption.sty with its "caption=false" package option. This is will preserve
% IEEEtran.cls handing of captions. Version 1.3 (2005/06/28) and later 
% (recommended due to many improvements over 1.2) of subfig.sty supports
% the caption=false option directly:
%\usepackage[caption=false,font=footnotesize]{subfig}
%
% The latest version and documentation can be obtained at:
% http://www.ctan.org/tex-archive/macros/latex/contrib/subfig/
% The latest version and documentation of caption.sty can be obtained at:
% http://www.ctan.org/tex-archive/macros/latex/contrib/caption/




% *** FLOAT PACKAGES ***
%
%\usepackage{fixltx2e}
% fixltx2e, the successor to the earlier fix2col.sty, was written by
% Frank Mittelbach and David Carlisle. This package corrects a few problems
% in the LaTeX2e kernel, the most notable of which is that in current
% LaTeX2e releases, the ordering of single and double column floats is not
% guaranteed to be preserved. Thus, an unpatched LaTeX2e can allow a
% single column figure to be placed prior to an earlier double column
% figure. The latest version and documentation can be found at:
% http://www.ctan.org/tex-archive/macros/latex/base/



%\usepackage{stfloats}
% stfloats.sty was written by Sigitas Tolusis. This package gives LaTeX2e
% the ability to do double column floats at the bottom of the page as well
% as the top. (e.g., "\begin{figure*}[!b]" is not normally possible in
% LaTeX2e). It also provides a command:
%\fnbelowfloat
% to enable the placement of footnotes below bottom floats (the standard
% LaTeX2e kernel puts them above bottom floats). This is an invasive package
% which rewrites many portions of the LaTeX2e float routines. It may not work
% with other packages that modify the LaTeX2e float routines. The latest
% version and documentation can be obtained at:
% http://www.ctan.org/tex-archive/macros/latex/contrib/sttools/
% Documentation is contained in the stfloats.sty comments as well as in the
% presfull.pdf file. Do not use the stfloats baselinefloat ability as IEEE
% does not allow \baselineskip to stretch. Authors submitting work to the
% IEEE should note that IEEE rarely uses double column equations and
% that authors should try to avoid such use. Do not be tempted to use the
% cuted.sty or midfloat.sty packages (also by Sigitas Tolusis) as IEEE does
% not format its papers in such ways.





% *** PDF, URL AND HYPERLINK PACKAGES ***
%
%\usepackage{url}
% url.sty was written by Donald Arseneau. It provides better support for
% handling and breaking URLs. url.sty is already installed on most LaTeX
% systems. The latest version can be obtained at:
% http://www.ctan.org/tex-archive/macros/latex/contrib/misc/
% Read the url.sty source comments for usage information. Basically,
% \url{my_url_here}.



\usepackage{code}
\usepackage{balance}
\usepackage{url}




% *** Do not adjust lengths that control margins, column widths, etc. ***
% *** Do not use packages that alter fonts (such as pslatex).         ***
% There should be no need to do such things with IEEEtran.cls V1.6 and later.
% (Unless specifically asked to do so by the journal or conference you plan
% to submit to, of course. )


% correct bad hyphenation here
\hyphenation{op-tical net-works semi-conduc-tor}


\begin{document}
%
% paper title
% can use linebreaks \\ within to get better formatting as desired
\title{Registered Report: First, Fuzz the Mutants}


\author{\IEEEauthorblockN{Alex Groce\\ and Goutamkumar Tulajappa Kalburgi}
\IEEEauthorblockA{Northern Arizona University\\
alex.groce@nau.edu, gk325@nau.edu}
\and
\IEEEauthorblockN{Claire Le Goues\\ and Kush Jain}
\IEEEauthorblockA{Carnegie Mellon University\\
clegoues@cs.cmu.edu, kdjain@andrew.cmu.edu}
\and
\IEEEauthorblockN{Rahul Gopinath}
\IEEEauthorblockA{CISPA, Saarland University\\
rahul@gopinath.org}}


% author names and affiliations
% use a multiple column layout for up to three different
% affiliations
%\author{\IEEEauthorblockN{Alex Groce\\ and Goutamkumar Tulajappa Kalburgi}
%\IEEEauthorblockA{Northern Arizona University\\
%alex.groce@nau.edu, gk325@nau.edu}
%\and
%\IEEEauthorblockN{Claire Le Goues\\ and Kush Jain}
%\IEEEauthorblockA{Carnegie Mellon University\\
%  clegoues@cs.cmu.edu, kdjain@andrew.cmu.edu}

%\and
%\IEEEauthorblockN{Rahul Gopinath}
%\IEEEauthorblockA{CISPA Saarland University\\
%  rahul@gopinath.org}

%}

% conference papers do not typically use \thanks and this command
% is locked out in conference mode. If really needed, such as for
% the acknowledgment of grants, issue a \IEEEoverridecommandlockouts
% after \documentclass

% for over three affiliations, or if they all won't fit within the width
% of the page, use this alternative format:
% 
%\author{\IEEEauthorblockN{Michael Shell\IEEEauthorrefmark{1},
%Homer Simpson\IEEEauthorrefmark{2},
%James Kirk\IEEEauthorrefmark{3}, 
%Montgomery Scott\IEEEauthorrefmark{3} and
%Eldon Tyrell\IEEEauthorrefmark{4}}
%\IEEEauthorblockA{\IEEEauthorrefmark{1}School of Electrical and Computer Engineering\\
%Georgia Institute of Technology,
%Atlanta, Georgia 30332--0250\\ Email: see http://www.michaelshell.org/contact.html}
%\IEEEauthorblockA{\IEEEauthorrefmark{2}Twentieth Century Fox, Springfield, USA\\
%Email: homer@thesimpsons.com}
%\IEEEauthorblockA{\IEEEauthorrefmark{3}Starfleet Academy, San Francisco, California 96678-2391\\
%Telephone: (800) 555--1212, Fax: (888) 555--1212}
%\IEEEauthorblockA{\IEEEauthorrefmark{4}Tyrell Inc., 123 Replicant Street, Los Angeles, California 90210--4321}}

% use for special paper notices
%\IEEEspecialpapernotice{(Invited Paper)}



\IEEEoverridecommandlockouts
\makeatletter\def\@IEEEpubidpullup{6.5\baselineskip}\makeatother
\IEEEpubid{\parbox{\columnwidth}{
     {\fontsize{8.5}{8.5}\selectfont International Fuzzing Workshop (FUZZING) 2022 \\
     27 February 2022, Virtual \\
     ISBN 1-891562-77-0 \\
     https://dx.doi.org/10.14722/fuzzing.2022.23xxx \\
     www.ndss-symposium.org}
}
\hspace{\columnsep}\makebox[\columnwidth]{}}


% make the title area
\maketitle



\begin{abstract}
Most fuzzing efforts, very understandably, focus on fuzzing the program
in which bugs are to be found.  However, in this paper we propose that
fuzzing programs ``near'' the System Under Test (SUT) can in fact
improve the effectiveness of fuzzing, even if it means less time is
spent fuzzing the actual target system.  In particular, we claim that
fault detection and code coverage can be improved by splitting fuzzing
resources between the SUT and \emph{mutants} of the SUT.  Spending
half of a fuzzing budget fuzzing mutants, and then using the seeds
generated to fuzz the SUT can allow a fuzzer to explore more behaviors
than spending the entire fuzzing budget on the SUT.  The approach
works because fuzzing most mutants is ``almost'' fuzzing the SUT, but
may change behavior in ways that allow a fuzzer to reach deeper
program behaviors.  Our results using Google's FuzzBench platform show that fuzzing mutants
is trivial to implement and fuzzer-agnostic, but provides clear, statistically significant
benefits in terms of branch coverage for a number of
real-world benchmarks, using AFLplusplus and Honggfuzz as baseline
fuzzers.  One of the variants of our method, using heuristically
chosen mutants,  ranks first by both of the two standard measures of fuzzer effectiveness
provided by FuzzBench.
\end{abstract}
% IEEEtran.cls defaults to using nonbold math in the Abstract.
% This preserves the distinction between vectors and scalars. However,
% if the conference you are submitting to favors bold math in the abstract,
% then you can use LaTeX's standard command \boldmath at the very start
% of the abstract to achieve this. Many IEEE journals/conferences frown on
% math in the abstract anyway.

% no keywords




% For peer review papers, you can put extra information on the cover
% page as needed:
% \ifCLASSOPTIONpeerreview
% \begin{center} \bfseries EDICS Category: 3-BBND \end{center}
% \fi
%
% For peerreview papers, this IEEEtran command inserts a page break and
% creates the second title. It will be ignored for other modes.
%%\IEEEpeerreviewmaketitle


\section{Introduction}


Consider the problem of fuzzing a program whose structure is as
follows:

\begin{code}
  if (!hard1(input)) \{
      return 0;
  \}
  if (!hard2(input)) \{
      return 0;
  \}
  crash();   
\end{code}

Assume that conditions {\tt hard1} and {\tt hard2} are independent
constraints on an input, both of which are difficult to achieve.  A
normal mutation-based
fuzzer such as AFL or libFuzzer attempting to reach the call to {\tt crash} will generally
first have to construct an input satsifying {\tt hard1} and then,
while preserving {\tt hard1}, modify that input until it also
satisfies {\tt hard2}.  A key point to note is that if the fuzzer
accidentally produces an input that is a good start on satisfying {\tt
  hard2}, or even completely satisfies {\tt hard2}, such an input will
be discarded, because execution never reaches the implementation of
{\tt hard2} unless {\tt hard1} has already been ``solved.''  Even
though the fuzzer must eventually satisfy both conditions, it can only
work on them in the execution order.  By analogy, consider the problem
of rolling a pair of \emph{ordered} dice.  If the goal is to roll two values above
five, and you are allowed to ``save'' a good roll of the first of the
two dice and use it in future attempts, the problem is easier than if
the dice have to be rolled from scratch each time.  However, it is not
as easy as if good rolls of the second die can also be saved, even if the first
die has never produced a five or six!

If we fuzz a program without the first {\tt return} statement:

\begin{code}
  if (!hard1(input)) \{
    /* return 0; */
  \}
  if (!hard2(input)) \{
      return 0;
  \}
  crash();   
\end{code}

\noindent then progress towards both {\tt hard1} and {\tt hard2} can
be made \emph{at the same time}, independently, in any order.  If a generated input progresses
achievement of either {\tt hard1} or {\tt hard2} it will be kept and
used in further fuzzing.   Of course, \emph{crashing inputs} for this
modified program are seldom crashing inputs for the
original program.  However, given a partial or total solution to {\tt
  hard1} and a partial or total solution to {\tt hard2}, it should be
much easier for a fuzzer to construct a crashing input for the
original program.  This is a very simple example of a case where
fuzzing a similar program can produce inputs such that 1) they help fuzz the
actual program under test and 2) those inputs are much harder, or
essentially impossible, to
generate for the original program using the fuzzer.

Three points are important to note about this approach: first, fuzzing an arbitrary program would be of no use here.  Inputs useful in exploring that program would likely be useless in exploring the real target of fuzzing.  Second, if a modification has little semantic impact on the original program, then fuzzing that variation is, to a large extent, the same as fuzzing the original program, with the only cost being some additional fuzzer logistics overhead.  Finally, predicting which program variants will aid fuzzing seems inherently hard.  In this case, removing a statemeent was extremely useful; in other cases breaking out of a loop before it fails a check might be important, or turning a condition into a constant true --- or constant false!  Analysis capable of detecting reliably ``good'' changes seems likely to be fundamentally about as hard as fuzzing itself, or symbolic execution.  Recall however, that many variants that are not useful will also be harmless, in that they amount to simply fuzzing the target.  What we need is a source of similar programs that will include the (perhaps rare) high-value variants (such as removing the {\tt return} above, and will not include too many programs so dis-similar in semantics they provide no value.

Program \emph{mutants} provide such variants, by design \cite{MutationSurvey}.  Mutants are designed to show weaknesses in a testing effort, by showing
the ability of a test suite to detect \emph{plausible bugs}.  The majority of such hypothetical bugs must be semantically similar enough to
the original program that a test suite's effectiveness is meaningful for the mutated program.  Therefore, most program mutants
will satisfy the condition of being close enough to the target of fuzzing.  Mutants are roughly evenly distributed over a program's source code,
and modify only a single location.  Therefore most uninteresting mutants will generally be harmless, since fuzzing the mutant will be essentially
fuzzing the original program, except for a small number of code paths.  Finally, mutation operators are varied enough to provide a good source
of potentially useful mutants.  Most importantly, almost all mutation tools include at least statement deletion (to remove checks that impede
fuzzing) and conditional changes (negation, and replacement with constant false and constant true).  These are the variants with the most
obvious potential for helping a fuzzer explore beyond a hard input constraint, as in the example above.

Additionally, fuzzing program mutants is a \emph{useful activity in itself}.  Mutation testing is increasingly being applied in the real-world.
A program worth fuzzing is probably a program worth examining from the perspective of mutation testing.  Examining mutants not detected by fuzzing
can reveal opportunities to improve a fuzzing effort, either by helping it reach hard-to-cover paths or, more frequently, by improving the oracle
(e.g., adding assertions about invariants a mutant causes to be violated, or even creating a new end-to-end fuzzing harness when a fault is
not exposed by fuzzing only isolated components of a program).  Mutation testing of the Bitcoin Core implementation (see the report
(\url{https://agroce.github.io/bitcoin_report.pdf}) referenced in the Bitcoin Core fuzzing documentation
(\url{https://github.com/bitcoin/bitcoin/blob/master/doc/fuzzing.md}) revealed just such limits to the fuzzing, despite its 
extremely high coverage and overall quality.  Mutation testing is supported by widely used and well-supported tools, and available for all
commonly used (and many uncommonly used) programming languages.


\section{Fuzzing the Mutants, in Detail}


\subsection{Mutation Testing}

Mutation
testing~\cite{MutationSurvey,budd1979mutation,demillo1978hints} is an
approach to evaluating and improving tests.  MOButation testing
introduces small syntactic changes into a program, under the
assumption that if the original program was correct, then a program
with slightly different semantics will be incorrect, and should be
detected by effective tests.  Mutation testing is used in software
engineering research, occasionally in industry at-scale, and in some
critical open-source work~\cite{mutKernel,mutGoogle,mutFacebook}.

A mutation testing approach is defined by a set of mutation operators.
Such operators vary widely in the literature, though a few, such as
deleting a small portion of code (such as a statement), negating a
conditonal, or replacing arithmetic and relational operations (e.g.,
changing {\tt +} to {\tt -} or {\tt ==} to {\tt <=}), are very widely
used.

For generating mutants, we use the Universal Mutator \cite{regexpMut}
(\url{https://github.com/agroce/universalmutator}), which provides a
wide variety of source-level mutants for almost any widely used
programming language, and has been used extensively to mutate C, C++,
Python, and Solidity code.

In principle, the ways in which mutants could be incorporated into a
fuzzing process are almost unlimited.  However, the basic approach can
be simplified by considering the fuzzing of mutants as a preparatory
stage for fuzzing the target, as in the introductory example.  The
simplest approach is to split a given time-budget for fuzzing in two.
First, fuzz the mutants.  Then, collect an input corpus from that
fuzzing, and fuzz the target program as usual, but for half the
desired time.

\subsection{Fuzzing: Two Key Decisions}

Given a set of all mutants of a target program, and a decision to
split a given fuzzing budget into a mutant-fuzzing stage followed by a
target-fuzzing stage, there are two major decisions to be made: how to
select a subset of mutants, and how to carry out fuzzing the chosen
mutants.

\subsubsection{Choosing the Mutants}

For most programs, reasonable (e.g., 24 hour) fuzzing budgets, and
approaches to fuzzing mutants discussed below, it is impossible to
fuzz all the mutants of the target program.  For instance, if a
program has a mere 1,000 lines of code, and 2,000 mutants (not an
implausible number), a 12 hour mutant fuzzing budget where each mutant
is fuzzed for five minutes only allows fuzzing of 144 mutants, less
than 1\% of the total mutants.  Two obvious options offer themselves:
purely random selection of mutants, under the assumption that we have
no simple way to predict the good mutants and that good mutants will
often be redundant.  For the second point, consider the example from
the introduction.  While less effective than removing the {\tt return}
statement, negating the condition, changing it to a constant false, or
modifying a constant return value inside {\tt hard1} may all allow
progress to be made on {\tt hard2} without first satisfying {\tt
  hard1}.  Other changes might relax the most difficult aspects of
{\tt hard1} allowing progress on the easier aspects of the condition,
and thus progress on {\tt hard2}.  Alternatively, while we cannot
predict the best mutants, it might be reasonable to try to diversify
the mutants selected using some kind of prioritization.  In
particular, in our recent work on using mutants to evaluated static
analysis tools \cite{QRS2021}, we proposed a scheme for ordering
mutants for humans to examine.  

The mutant prioritization
uses Gonzalez' Furthest-Point-First \cite{Gonzalez} (FPF) algorithm
to \emph{rank} mutants, as earlier work had used it to rank test cases for identifying faults \cite{PLDI13}.
An
FPF ranking requires a distance metric $d$, and ranks items so that
dissimilar ones appear earlier.  FPF is a
greedy algorithm that proceeds by repeatedly adding the item with the
\emph{maximum minimum distance to all previously ranked items}. Given an
initial seed item $r_0$, a set $S$ of items to rank, and a distance
metric $d$, FPF computes $r_i$ as
$s \in S: \forall s' \in S: min_{ j < i}(d(s,r_j)) \geq min_{j <
  i}(d(s',r_j))$.  The condition on $s$ is obviously true when
$s = s'$, or when $s' = r_j$ for some $j < i$; the other cases for
$s'$ force selection of \emph{some}
max-min-distance $s$.


The Universal Mutator \cite{regexpMut} tool's FPF metric $d$ is
the sum of a set of measurements.  First, it adds a similarity
ratio based on Levenshtein distance \cite{lev} for (1) the \emph{changes} (Levenshtein edits) from
the original source code elements to
the two mutants,  (2) the two original source code elements changed (in
general, lines), and (3) the actual output mutant code.  These are
weighted with multipliers of 5.0, 0.1, and 0.1, respectively; the type
of change (mutation operator, roughly) dominates this part of the
distance, because it best describes ``what the mutant did''; however,
because many mutants will have the same change (e.g., changing {\tt +}
to {\tt -}, the other values decide many cases.
%The Python
%{\tt Levenshtein} library's similarity ratio is used, as it is based
%on true minimal string edits; it reports similarity ratios between 0.0
%and 1.0.
The metric also incorporates the distance in the source
code between the locations of two mutants.  If the mutants are to
different files, this adds 0.5; it also adds 0.25
times the number of source lines separating the two mutants if they
are in the same file, divided by 10, but caps the amount added at
0.25.  The full metric, therefore is:

$$ 5.0 \times r(\mathit{edit}_1, \mathit{edit}_2) + 0.1 \times r(\mathit{source}_1, \mathit{source}_2) +$$
$$0.1 \times r(\mathit{mutant}_1, \mathit{mutant}_2) + 0.5 \times \mathit{not\_same\_file} +$$
$$max(0.25, \frac{\mathit{line\_dist}(\mathit{mutant}_1, \mathit{mutant}_2)}{10})$$

\noindent Where $r$ is a Levenshtein-based string similarity ratio,
$\mathit{line\_dist}$ is the distance in a source file between
two locations, in lines (zero if the locations are in different
files), and $\mathit{not\_same\_file}$ is 0/1.

\subsubsection{Using the Mutants}

The second key choice is how to use the chosen mutants.  Assuming a fixed budget per mutant, the most
basic choice is whether to fuzz each mutant ``from scratch'' (presumably using any existing corpus for
fuzzing the target), which we call non-cumulative/parallel fuzzing,  or to use each mutant's output corpus to seed the next mutant, which we call cumulative/sequential fuzzing.  The cumulative/sequential
approach has two potential advantages:

\begin{enumerate}
\item Mutants have some of the benefits of fuzzing with mutants, so hitting a key location that has been
mutated may be more likely.
\item The final corpus from the last-fuzzed mutant will contain few redundancies, reducing processing or
fuzzer startup time for the target.
\end{enumerate}

On the other hand, it forces processing of the corpus after each
mutant to remove inputs causing the next mutant to crash, and, more
importantly, prevents fuzzing mutants in parallel.  The processing
cost is due to the fact that before fuzzing a mutant or the target,
any input corpus needs, for the AFL fuzzer at least, to be pruned, removing any
crashing inputs that did not crash the previous mutant.  These should
be kept, as they may represent uniquely detected faults.  Removing these sequentially, rather than in a single batch after all mutants, may remove inputs that could have been useful for some mutant they do not crash, but re-trying all inputs for each mutant is expensive.


\section{Related Work}

Given that getting past verification checks is one of the most common problems in fuzzing,
(manually disabling verification checks is one of the most common proposals in practical \cite{chromeadvice}
suggestions on improve the effectiveness of fuzzing)
numerous previous researchers have tried to bypass such checks by
patching the program itself.
An early attempt to do this was Flayer \cite{drewry2O07flayer} which
provides a mechanism for instrumenting the program, altering the control flow,
and stepping over function calls. The research also introduces a complementary
fuzzer that makes use of Flayer for more effective fuzzing.

A similar approach was taken in TaintScope \cite{wang2010taintscope},
which claims to be the first \emph{checksum-aware} fuzzer.
 It detects checksum-based integrity verification using branch profiling, and once found, it can
bypass such checks by altering the control flow.

CAFA \cite{liu2018cafa} is another fuzzer that uses taint analysis to detect the
parts of the program that are involved in checksum-based verification of
input integrity. Once detected, it statically patches the program to bypass
checksum verification of the input.

The most closely related work is the T-Fuzz approach \cite{tfuzz}, which focused specifically on removing sanity checks in programs in order to fuzz more deeply.  Our approach is motivated in part by the desire to remove sanity checks, but uses a more general and lightweight approach.  T-Fuzz used dynamic analysis to identify sanity checks, while we simply trust that program mutants will include many (or most) sanity checks.  Moreover, when a sanity check is hard to identify, but implemented by a function call, statement deletion mutants may in effect remove it where T-Fuzz will not.  Our approach also introduces changes that are not within the domain of T-Fuzz or the other fuzzers discussed above, e.g., changing conditions to include one-off values.  Finally, T-Fuzz worked around the fact that inputs for the modified program are not inputs for the real program under test using a symbolic execution step, while we simply hand the inputs generated for mutants to a fuzzer and trust a good fuzzer to make use of these ``hints'' to find inputs for the real program, if they are close enough to be useful.

Mutation analysis has been used previously to detect anomalies in programs
statically \cite{arcaini2017novel}. As in our approach, the program variants
are produced using mutation analysis, but the idea here is to look for variants
that are semantically equivalent, but better in some specific sense than the
original.

Arguably, UBSAN is a program transformer that explicitly doesn't preserve all the program
semantics (only the explicitly defined language semantics are preserved), and can improve fuzzing effectiveness.
It detects undefined behavior by inserting crashes when such behavior is invoked.

Finally, mutants may prove to be effective against anti-fuzzing \cite{jung2019fuzzification}
techniques such as speed-bumps (a mutant could either remove the bump or simply decrease delay/wait loop parameters).


\section{Proposed Evaluation}

In a full experimental evaluation, we will undertake to answer the following core research questions:

\begin{itemize}
  \item {\bf RQ1}: Does replacing time spent fuzzing a target program with time spent fuzzing mutants of the target program improve
  the effectiveness of fuzzing?
  \item {\bf RQ2:} Does using prioritization improve the effectiveness of fuzzing with mutants?
  \item {\bf RQ3:} How do non-cumulative (parallel) and cumulative (sequential) mutant fuzzing compare?
  \end{itemize}
  
{\bf RQ1} is the overall question of whether any variant of fuzzing using mutants increases standard fuzzing evaluation metrics
(unique faults detected and code coverage).  {\bf RQ2} and {\bf RQ3} consider some of the primary choices to be made in implementing fuzzing mutants.

The experiments will be based on widely-used benchmarks, and conform to the standards proposed by Klees et. al \cite{evalfuzz}, e.g., using 10 or
more runs of 24 hours each in experimental trials.  One simplifying factor in experiments on this question is that, since the approach concerns
only the choice of fuzzing targets and seeds, a single widely-used fuzzer such as the latest version of AFL, is justified.  It seems clear that
the advantages provided by fuzzing mutants should be orthogonal to the varying features of AFL, AFLPlusPlus, libFuzzer, and other commonly
used fuzzers.



\begin{table*}
  \renewcommand{\arraystretch}{1.3}
\caption{Results for preliminary experiments}
  
  \centering
  \begin{tabular}{l||r|r|r||r|r|r||r|r|r}
    & \multicolumn{3}{|c||}{Distinct Faults} & \multicolumn{3}{|c||}{Statement Coverage} &
                                                                    \multicolumn{3}{|c}{Branch Coverage} \\
    \hline
  Method & Min & Max  & Avg & Min & Max & Avg
                                                                  
  & Min & Max & Avg \\
    \hline
    \hline
   \multicolumn{10}{c}{Baseline (no mutants)} \\    
    \hline
  AFL on program only & 3 & 5 & 4.2 & 79.86\% & 84.37\% & 81.73\% &
                                                                    78.36\%
                                  & 81.35\% & 80.40\%\\
    \hline
    \hline
    \multicolumn{10}{c}{AFL on random mutants} \\
    \hline
 Non-cumulative  & {\bf 6} & {\bf 7} & {\bf 6.4} & 80.04\% &
                                                                   {\bf
                                                                                     84.90\%}
                      & 81.70\% & 79.85\% & {\bf 82.58\%} & 80.70\%\\
  \hline
 Cumulative/sequential & {\bf 6} & {\bf 7} & 6.2 & 80.21\%
                                                               &
                                                                 {\bf 84.90\%}
                      & 81.77\%
                            & 80.10\% & 82.34\% & 80.90\%\\
    \hline
    \hline
    \multicolumn{10}{c}{AFL on prioritized mutants} \\
    \hline
    Non-cumulative  & {\bf 6} & {\bf 7} & 6.2 &
                                                                {\bf 81.25\%}
               & 84.37\% & {\bf 82.39\%} & {\bf 80.60\%} & 81.84\% & 81.20\% \\
    \hline
   Cumulative/sequential  & {\bf 6} &
                                                                   {\bf 7} & 6.2 &
    {\bf 81.25\%} & {\bf 84.90\%} & 83.16\% & 80.10\% & {\bf 82.58\%} & {\bf 81.39\%}\\    
  \hline
  \end{tabular}
  \label{tab:prelim}

\end{table*}

\section{Preliminary Experiments}



Table \ref{tab:prelim} shows results of fuzzing the {\tt fuzzgoat}
(\url{https://github.com/fuzzstati0n/fuzzgoat}) benchmark program for
fuzzers, with and without using mutants to aid the fuzzing.  We
applied our basic technique, using both random and prioritized (by
Universal Mutator) mutant selection, and using non-cumulative and
cumulative mutant fuzzing.  For non-cumulative mutant fuzzing, we did
not perform an initial stage of fuzzing on the target program. The
best value(s) for each evaluation measure are highlighted in bold.


Each
technique evaluated was used in 5 fuzzing attempts of 10 hours each.  The
baseline for comparison is the latest Google release of  AFL (2.57b) on the {\tt fuzzgoat} program for 10
hours, with no time spent in any effort other than fuzzing {\tt
  fuzzgoat}.  The other approaches apply the basic methods for using
mutants described above, for five hours, then fuzz using the resulting
corpus for another five hours.  These approaches all spend a small
fraction of the fuzzing budget restarting AFL and processing
already-generated inputs (e.g., to make sure they don't crash the
original program, even if they did not crash a mutant), rather than
fuzzing either {\tt fuzzgoat} or a mutant.  The budget for fuzzing
each mutant is fixed at five minutes, so only about 60 of the nearly
3,800 mutants of {\tt fuzzgoat.c} can be fuzzed.  For the first two
mutant runs, these mutants were chosen randomly each time; the second
two runs used a fixed set of mutants, based on the default mutant
prioritization scheme provided by the Universal Mutator, with the option to
prioritize all statement deletions above other mutants set to false.  Coverage was measured using {\tt gcov} and faults were determined by using address sanitizer to determine locations of memory access violations, and examining the traces to determine the distinct faults.

Fault
detection was \emph{uniformly better} for all mutant-based approaches than
for fuzzing without mutants; the minimum number of detected faults was
better than the maximum number of faults found without using mutants.
Fault detection partly benefited from crashes detected only during fuzzing of mutants.
However, even ignoring these crashes, three of the mutant-based efforts detected six distinct faults, while fuzzing without mutants never detected six faults.  Means for the techniques without using crashes discovered during mutant fuzzing were, respectively (in the same order as the table): 4.8, 4.6, 5.0, and 5.0, still all higher than for fuzzing without mutants.  Using the crashes from mutant fuzzing, every mutant-based effort detected all vulnerabilities in fuzzgoat of which we are aware.


Code coverage results were more ambiguous, but the limited data
suggests the prioritized mutant approaches may be more consistent in
hitting hard-to-reach code than the other methods.  In particular, the highest branch coverage numbers were all reached by prioritized mutant fuzzing, and the worst statement and branch coverage values were from fuzzing without mutants.

Coverage differences were not statistically significant by Mann
Whitney U test, but bug count differences between all mutant-based methods and AFL
without mutants were significant with $p$-value $< 0.006$.
Differences in unique faults detected were not significant, when
faults detected only during mutant fuzzing were discarded (though this
is likely only due to the small sample size and range of values;
p-values were around 0.2).

While it is clear that for this benchmark program, fuzzing mutants
provides an advantage, it is also clear that distinguishing between
variations of the basic approach is not possible without considerably
more experimental data across more subjects.

Finally, we note that our experiments support our claim that the proposed technique is almost trivial to apply. We were able to implement mutant fuzzing in less than 30 lines of Python, and replacing AFL with another fuzzer would be trivial.






\section{Conclusions and Future Work}

In this paper we propose that by fuzzing variations of a target program generated by a mutation testing tool, it may be possible to work around some fundamental limitations of coverage-driven fuzzing.  For the most part, even when not effective, the technique proposed should be low-cost and at worst equivalent to fuzzing the target program itself for a somewhat smaller time.  Our preliminary experiments show that fuzzing mutants is trivial to implement (and applies to any fuzzer of which we are aware) and effective for improving fault detection, at least for a non-trivial benchmark target program.

Future work, in addition to the performance of full experiments to evaluate the technique, would include exploring the effectiveness of using mutation selection methods and prioritization techniques in addition to those proposed here, and applying directed greybox fuzzing \cite{AFLGo} to specifically target mutated code.  Another possibility is to use Higher Order Mutants \cite{HOM} to fuzz multiple mutants at once; however, this increases the chance that a critical mutant will be combined with a mutant that essentially destroys the program semantics, making it impossible to exploit.


% can use a bibliography generated by BibTeX as a .bbl file
% BibTeX documentation can be easily obtained at:
% http://www.ctan.org/tex-archive/biblio/bibtex/contrib/doc/
% The IEEEtran BibTeX style support page is at:
% http://www.michaelshell.org/tex/ieeetran/bibtex/
%\bibliographystyle{IEEEtranS}
% argument is your BibTeX string definitions and bibliography database(s)
%\bibliography{IEEEabrv,../bib/paper}
%
% <OR> manually copy in the resultant .bbl file
% set second argument of \begin to the number of references
%   (used to reserve space for the reference number labels box)

\balance

\bibliographystyle{IEEEtranS}
\bibliography{bibliography}





% that's all folks
\end{document}


