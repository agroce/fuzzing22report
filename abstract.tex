Most fuzzing efforts, very understandably, focus on fuzzing the program
in which bugs are to be found.  However, in this paper we propose that
fuzzing programs ``near'' the System Under Test (SUT) can in fact
improve the effectivness of fuzzing, even if it means less time is
spent fuzzing the actual target system.  In particular, we claim that
fault detection and code coverage can be improved by splitting fuzzing
resources between the SUT and \emph{mutants} of the SUT.  Spending
half of a fuzzing budget fuzzing mutants, and then using the seeds
generated to fuzz the SUT can allow a fuzzer to explore more behaviors
than spending the entire fuzzing budget on the SUT.  The approach
works because fuzzing most mutants is ``almost'' fuzzing the SUT, but
may change behavior in ways that allow a fuzzer to reach deeper
program behaviors.  This approach has two additional important advantages:
first, it is fuzzer-agnostic, applicable to any corpus-based fuzzer
without requiring modification of the fuzzer; second, the fuzzing of
mutants, in addition to aiding fuzzing the SUT, also gives developers
insight into the mutation score of a fuzzing harness, which may help
guide improvements to a project's fuzzing approach.