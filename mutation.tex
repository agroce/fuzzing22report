\section{Fuzzing the Mutants, in Detail}


\subsection{Mutation Testing}

Mutation testing~\cite{MutationSurvey,budd1979mutation,demillo1978hints} is an approach to evaluating and improving tests.  Mutation testing introduces small syntactic changes into a program, under the assumption that if the original program was correct, then a program with slightly different semantics will be incorrect, and should be detected by effective tests.  Mutation testing is used in software engineering research, occasionally in industry at-scale, and in some critical  open-source work~\cite{mutKernel,mutGoogle,mutFacebook}.

A mutation testing approach is defined by a set of mutation operators.  Such operators vary widely in the literature, though a few, such as deleting a small portion of code (such as a statement), negating a conditonal, or replacing arithmetic and relational operations (e.g., changing {\tt +} to {\tt -} or {\tt ==} to {\tt <=}), are very widely used.