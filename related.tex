\section{Related Work}

The most closely related work is the the T-Fuzz approach~\cite{tfuzz}, which focused specifically on removing sanity checks in programs in order to fuzz more deeply.  Our approach is motivated in part by the desire to remove sanity checks, but uses a more general and lightweight approach.  T-Fuzz used dynamic analsyis to identify sanity checks, while we simply trust that program mutants will include many (or most) sanity checks.  Moreover, when a sanity check is hard to identify, but implemented by a function call, statement deletion mutants may in effect remove it where T-Fuzz will not.  Our approach also introduces changes that are not within the domain of T-Fuzz, e.g., changing conditions to include one-off values.  Finally, T-Fuzz worked around the fact that inputs for the modified program are not inputs for the real program under test using a symbolic exeuction step, while we simply hand the inputs generated for mutants to a fuzzer and trust a good fuzzer to make use of these ``hints'' to find inputs for the real program, if they are close enough to be useful.